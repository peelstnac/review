\documentclass[11pt]{scrartcl}
\usepackage[sexy]{evan}
\usepackage{quiver}
\usepackage{adjustbox}


\begin{document}
\title{Solutions}

\maketitle

\begin{problem*}[4-3]
\end{problem*}

For notation, take $\lra{\cdot, \cdot}$ to be $T(\cdot, \cdot)$.
Let $e_1, ..., e_n$ be an orthonormal basis for $V$.
By definition, $|\omega(e_1, ...., e_n)| = 1$, and applying Theorem 4-6 gives 
$$|\omega(e_1, ..., e_n)| = \left|\det\sedm{\lra{w_i, e_j}}\right|.$$
Using the common identity (see any quantum mechanics textbook)
$$\sedm{\lra{w_i, e_j}}\sedm{\lra{e_i, w_j}} = \sedm{\lra{w_i, w_j}},$$ 
we arrive at the desired result.

\begin{problem*}[4-4]
\end{problem*}

\begin{align*}
    f^*\omega(v_1, ..., v_n) 
    &= 
    \omega(f(v_1), ..., f(v_n)) \\
    &=
    \det\sedm{T(f(v_i), f(e_j))}\omega(f(e_1), ..., f(e_n)) &\text{(Theorem 4-3)} \\
    &=
    \det\sedm{T(f(v_i), f(e_j))}\\
    &= 
    \det\sedm{\lra{v_i, e_j}},
\end{align*}
where the last two equalities come from the fact that $\omega$ is a volume element and $f^*T(\cdot, \cdot) = \lra{\cdot, \cdot}$.

\begin{problem*}[4-5]
\end{problem*}
Because $\det$ is continuous, the image of $\det\circ c$ on the path must be of the same sign.

\begin{problem*}[4-6]
\end{problem*}


\end{document}