\documentclass[11pt]{scrartcl}
\usepackage[sexy]{evan}
\usepackage{quiver}
\usepackage{adjustbox}


\begin{document}
\title{Solutions}

\maketitle

\begin{problem*}[4-3]
\end{problem*}

For notation, take $\lra{\cdot, \cdot}$ to be $T(\cdot, \cdot)$.
Let $e_1, ..., e_n$ be an orthonormal basis for $V$.
By definition, $|\omega(e_1, ...., e_n)| = 1$, and applying Theorem 4-6 gives 
$$|\omega(e_1, ..., e_n)| = \left|\det\sedm{\lra{w_i, e_j}}\right|.$$
Using the common identity (see any quantum mechanics textbook)
$$\sedm{\lra{w_i, e_j}}\sedm{\lra{e_i, w_j}} = \sedm{\lra{w_i, w_j}},$$ 
we arrive at the desired result.

\begin{problem*}[4-4]
\end{problem*}

\begin{align*}
    f^*\omega(v_1, ..., v_n) 
    &= 
    \omega(f(v_1), ..., f(v_n)) \\
    &=
    \det\sedm{T(f(v_i), f(e_j))}\omega(f(e_1), ..., f(e_n)) &\text{(Theorem 4-3)} \\
    &=
    \det\sedm{T(f(v_i), f(e_j))}\\
    &= 
    \det\sedm{\lra{v_i, e_j}},
\end{align*}
where the last two equalities come from the fact that $\omega$ is a volume element and $f^*T(\cdot, \cdot) = \lra{\cdot, \cdot}$.

\begin{problem*}[4-5]
\end{problem*}
Because $\det$ is continuous, the image of $\det\circ c$ on the path must be of the same sign.

\begin{problem*}[4-6]
\end{problem*}
(a) We have $v_1\times v_2 = \det[v_1\ v_2]$.

(b) By definition, $\det[v_1, ..., v_{n - 1}, v_1\times\cdots \times v_{n - 1}] = ||v_1\times\cdots \times v_{n - 1}||^2$.

\begin{problem*}[4-7]
\end{problem*}
Fix $\omega\in \wedge^n(V)$.
Let $S$ be any inner product on $V$.
Construct inner product $T$ by scaling $S$ by $\frac{1}{\omega(e_1, ..., e_n)^2}$, 
where $e_1, ..., e_n$ is an orthonormal basis for $V$.
Now, it's easy to see that $\omega$ is a volume element with respect to $T$.

\begin{problem*}[4-8]
\end{problem*}
Take $T$ to be the inner product on $V$.
Let $\varphi\in V^*$ be given by $\varphi(v) = \omega(v_1, ..., v_{n - 1}, v)$, 
where $v_1, ..., v_{n - 1}$ are fixed.
Then, by the Riesz representation theorem, $\varphi(v) = T(v, w)$ for some $w\in V$.
We can now define $v_1\times \cdots \times v_{n - 1} = w$.

\begin{problem*}[4-9]
\end{problem*}
(a) and (b) By definition, 
$$
    \lra{v\times w, e_i} = \det[v \ w\ e_i],
$$
and the formula in (b) follows.
(a) is obtained by applying (b).

(c) We have 
\begin{align*}
    \frac{1}{\norm{v}^2\norm{w}^2}\norm{v\times w}^2
    &=
    \frac{1}{\norm{v}^2\norm{w}^2}
    \norm{\vv{
        v_2w_3 - v_3w_2 \\
        v_3w_1 - v_1w_3 \\
        v_1w_2 - v_2w_1
    }}^2 \\
    &=
    \frac{1}{\norm{v}^2\norm{w}^2}
    (
        v_2^2w_3^2 + v_3^2w_2^2 - 2v_2v_3w_2w_3 \\
        &+
        v_3^2w_1^2 + v_1^2w_3^2 - 2v_1v_3w_1w_3 \\
        &+
        v_1^2w_2^2 + v_2^2w_1^2 - 2v_1v_2w_1w_2
    ) \\
    &= 1 - \frac{\lra{v, w}^2}{\norm{v}^2\norm{w}^2} \\
    &= 1 - \cos^2\theta \\
    &= \sin^2\theta
\end{align*}
so $\norm{v\times w} = \norm{v}\norm{w}\sin\theta$.
By definition, 
$$
    \lra{v\times w, v} = \det[v\ w\ v] = 0,
$$
and $\lra{v\times w, w} = 0$ follows similarly.

(d) We have
$$
    \lra{v, w\times z} = \det[w\ z\ v] = \lra{w, z\times v} = \det[z\ v\ w] = \lra{z, v\times w} = \det[v\ w\ z].
$$
To prove the vector triple product identity, we have to bash out some algebra.
I'm too lazy to type it out, so I will leave it as an exercise to the reader.

(e) We have 
\begin{align*}
    \norm{v\times w}^2 
    &=
    \norm{v}^2\norm{w}^2\sin^2\theta \\
    &=
    \norm{v}^2\norm{w}^2(1 - \cos^2\theta) \\
    &=
    \norm{v}^2\norm{w}^2 - \lra{v, w}^2 \\
    &=
    \lra{v, v}\lra{w, w} - \lra{v, w}^2.
\end{align*}

\begin{problem*}[4-10]
\end{problem*}
By definition, 
$$
    \norm{w_n}^2 = \det\vv{w_1\\ \vdots \\ w_{n - 1}\\ w_n},
$$
where $w_n = w_1\times\cdots\times w_{n - 1}$.
Observe that $w_1, ..., w_{n - 1}\in \spn{w_n}^\perp$, since 
$$
    \lra{w_i, w_n} = \det\vv{w_1\\ \vdots\\ w_{n - 1}\\ w_i} = 0
$$ 
if $1\le i\le n - 1$.

By problem 4-3, we have 
\begin{align*}
    \det\vv{w_1\\ \vdots \\ w_{n - 1}\\ w_n} 
    &= 
    \sqrt{\det[\lra{w_i, w_j}]_{1\le i, j\le n}} \\
    &= 
    \sqrt{\lra{w_n, w_n}\det[\lra{w_i, w_j}]_{1\le i, j\le n - 1}} \\
    &=
    \norm{w_n}\sqrt{\det[\lra{w_i, w_j}]_{1\le i, j\le n - 1}}.
\end{align*}
Thus, $\norm{w_n} = \sqrt{\det[\lra{w_i, w_j}]_{1\le i, j\le n - 1}}$.

\begin{problem*}[4-11]
\end{problem*}
Since $v_1, ..., v_n$ is an orthonormal basis, this result follows by the definition, 
since $a_{i, j} = \lra{f(v_i), v_j} = \lra{f(v_j), v_i} = a_{j, i}$.

\begin{problem*}[4-13]
\end{problem*}
(a) We have 
\begin{align*}
    (g\circ f)_*(v_p) 
    &= 
    ((g\circ f)'(p)v)_{g(f(p))}\\
    &=
    (g'(f(p))f'(p)v)_{g(f(p))} \\
    &=
    g_*(f'(p)v_{f(p)}) \\
    &=
    g_*(f_*(v_p)).
\end{align*}
Let $\omega$ be a $k$-form on $\RR^p$.
Then,
\begin{align*}
    ((g\circ f)^*\omega)(p)(v_1, ..., v_k) 
    &=
    \omega(g(f(p)))(g_*(f_*(v_1)), ..., g_*(f_*(v_k))) \\
    &=
    (g^*\omega)(f(p))(f_*(v_1), ..., f_*(v_k)) \\
    &=
    (f^*g^*\omega)(p)(v_1, ..., v_k).
\end{align*}

(b) Follows from Theorem 4-7.

\begin{problem*}[4-14]
\end{problem*}
Follows immediately from problem 4-13.

\begin{problem*}[4-15]
\end{problem*}
Tangent vector is $c_*((e_1)_t) = (1, f'(t))_{c(t)}$.
Endpoint is $(1 + t, f'(t) + f(t))$.
Tangent line $y = f'(t)(x - t) + f(t)$.
One may verify that the endpoint lies on the tangent line.

\begin{problem*}[4-16]
\end{problem*}
We have $\frac{d}{dt}\norm{c(t)}^2 = 0 = 2\lra{c(t), c'(t)}$.

\begin{problem*}[4-17]
\end{problem*}
(a) Trivial.
(b) $\nabla\cdot \mathbf{f} = \sum_i\frac{\partial}{\partial x^i}f^i = \Tr{f'}$ by definition.

\begin{problem*}[4-18]
\end{problem*}
$D_vf(p) = \frac{d}{dt}f(p + tv)|_{t = 0} = f'(p)v$ via chain rule.
The rest follows from Cauchy-Schwarz.

\begin{problem*}[4-19]
\end{problem*}
(a)
$df = \sum_i\frac{\partial}{\partial x^i}f^idx^i = \omega^1_{\nabla f}$.
\begin{align*}
    d\omega^1_F 
    &=
    dF^1 dx + dF^2 dy + dF^3 dz \\
    &=
    \pd{y}F^1 dy\wedge dx + \pd{z}F^1 dz\wedge dx + \pd{x}F^2 dx\wedge dy + \pd{z}F^2 dz\wedge dy + \pd{x}F^3 dx\wedge dz + \pd{y}F^3 dy\wedge dz \\
    &=
    \left(\pd{x}F^2 - \pd{y}F^1\right)dx\wedge dy + \left(\pd{y}F^3 - \pd{z}F^2\right)dy\wedge dz + \left(\pd{z}F^1 - \pd{x}F^3\right)dz\wedge dx \\
    &=
    \omega^2_{\nabla \times F}.
\end{align*}

\begin{align*}
    d\omega^2_F 
    &= 
    dF^1 dy\wedge dz + dF^2 dz\wedge dx + dF^3 dx\wedge dy \\
    &=
    \pd{x}F^1 dx\wedge dy\wedge dz + \pd{y}F^2 dy\wedge dz\wedge dx + \pd{z}F^3 dz\wedge dx\wedge dy \\
    &=
    \left(\pd{x}F^1 + \pd{y}F^2 + \pd{z}F^3\right)dx\wedge dy\wedge dz \\
    &=
    \nabla\cdot F dx\wedge dy\wedge dz.
\end{align*}

(b) We have $d^2f = 0 = \omega^2_{\nabla\times\nabla f}$.
We have $d^2\omega^1_{\nabla f} = 0 = \nabla\cdot f dx\wedge dy\wedge dz$.

(c) Since $\nabla\times F = 0, d\omega^1_F = 0$.
By Poincare's lemma, there exist a function $f$ such that $df = \omega^1_F$.
We have $df = \omega^2_{\nabla f}$, so $\nabla f = F$.

\end{document}
